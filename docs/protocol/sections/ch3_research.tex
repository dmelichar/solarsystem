\section{Recherche und evaluierung}
\label{sec:research}

\subsection{Python OpenGL Libraries}

Folgende libaries für die Umsetzung der Applikation wurden evauliert.

\begin{itemize}
	\item Pygame
	\item Pyglet
	\item Panda3D
\end{itemize}

Letzendlich haben wir uns für Pyglet entschieden, da dieses die beste funktionalität bietet mit den geringsten
Installationsaufwand. Pand3D besitzt bereits ein fertiges Example für ein Solar System, was die Sache ein wenig
zu einfach machen würde. Pygame ist im gegensatz zu Pyglet nicht mehr so gut maintained und benötigt noch
zusätzliche libraries um z.B. texturen zu setzten. 
\\
Für die GUI der Applikation hat sich Pyglet-GUI angeboten. Es verfügt zwar zugegeben nicht über all zu viele
Features, aber über genügend um ein simples Interface zu erstellen.

\subsection{Mathematische funktionen von elliptischen Bahnen}
Für diese wurden die in den Quellen angegeben Berechnungen verwendet. Diese Quellen wurden von Hrn. Rene Hollander
vorgeschlagen. Die Umsetzung der Bahnen war eigentlich nur ein Umsetzen der Gleichungen.