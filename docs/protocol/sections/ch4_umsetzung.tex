\section{Umsetzung der Applikation}
\label{sec:umsetzung}

\subsection{Erste Phase: Prototyping}
In dieser Phase ging es prinzipiell darum ein wenig Ahnung von Pyglet und OpenGL im
allgemeinen zu erhalten. Wir haben hier zunächst eine einfache \textit{Hello World} Applikation
mit Pyglet erstellt - hierbei gab es kaum probleme.
\\ \\
Beim nächsten Prototyp ging es darum einen konkreten Planeten zu erstellen, das heißt mit Textur, 
mit genauer größe und um eine Axe rotierend. Dies auch relativ einfach, und mit Verwendung der
\textit{Noise} library sah es sogar sehr Anspruchsvoll aus. 
\\ \\
Der dritte Prototyp beschäftigte sich damit, vertraut mit Pyglet-GUI zu werden. Es wurde hierfür
eine einfache GUI mit einigen Buttons erstellt, und auch gleich evaluiert welche möglichkeiten
durch pyglet-gui geboten werden.

\subsection{Zweite Phase: Recherche}
Siehe Recherche und evaluierung.

\subsection{Dritte Phase: GUI- und technisches Design}
Siehe Architektur- und UI-Design.

\subsection{Vierte Phase: Konkrete Umsetzung}
Bei der konkreten Umsetzung war das erste Ziel das UML in Klassen umzuwandeln. Dies ist auch relativ
gut gegangen. Und bei der ersten Ausführung waren auch die Anforderungen umgesetzt - sprich ein
zentraler Stern und ein darum rotierender Planet mit Mond. 
\\ \\
Nun ging es darum das noch ein wenig schöner und Objektorientierter zu gestallten. Dafür wurde eine
eigenes Modul \textit{util} für die Kamera, die Skybox (alle Planeten und generell astronomische Objekte befinden
sich in einer Sphere welche eine Textur hat), eine Klasse die Texturen managed (also korrekte größen und dergleichen),
und einen FPSCounter welcher die FPS der Applikation anzeigt.

\subsection{Aufgetretete Fehler}
Während der Entwicklung sind ein paar Fehler aufgetreten, diese wurden aber alle behoben bis auf einen. Dieser eine
Fehler ist der Grund warum die GUI nicht ganz korrekt funtkioniert. Wir haben bereits öfters versucht (mittels Debugging)
diesen Fehler zu beheben, leider ist uns das nicht gelungen - die Funktionen sind aber gegeben.