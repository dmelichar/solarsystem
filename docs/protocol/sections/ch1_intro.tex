\section{Aufgabenstellung}
\label{sec:Introduction}

In einem Team (2) sind folgende Anforderungen zu erfüllen.
\begin{itemize}
\item Ein zentraler Stern
\item Zumindest 2 Planeten, die sich um die eigene Achse und in elliptischen Bahnen um den Zentralstern drehen
\item Ein Planet hat zumindest einen Mond, der sich zusätzlich um seinen Planeten bewegt
\item Kreativität ist gefragt: Weitere Planeten, Asteroiden, Galaxien,...
\item Zumindest ein Planet wird mit einer Textur belegt (Erde, Mars,... sind im Netz verfügbar)
\end{itemize}

\textbf{Events}
\begin{itemize}
\item Mittels Maus kann die Kameraposition angepasst werden: Zumindest eine Überkopf-Sicht und parallel der Planentenbahnen
\item Da es sich um eine Animation handelt, kann diese auch gestoppt werden. Mittels Tasten kann die Geschwindigkeit gedrosselt und beschleunigt werden.
\item Mittels Mausklick kann eine Punktlichtquelle und die Textierung ein- und ausgeschaltet werden.
\item Schatten: Auch Monde und Planeten werfen Schatten.
\end{itemize}

\textbf{Hinweise}
\begin{itemize}
\item Ein Objekt kann einfach mittels glutSolidSphere() erstellt werden.
\item Die Planten werden mittels Modelkommandos bewegt: glRotate(), glTranslate()
\item Die Kameraposition wird mittels gluLookAt() gesetzt
\item Bedenken Sie bei der Perspektive, dass entfernte Objekte kleiner - nahe entsprechende größer darzustellen sind.
\item Wichtig ist dabei auch eine möglichst glaubhafte Darstellung. gluPerspective(), glFrustum()
\item Für das Einbetten einer Textur wird die Library Pillow benötigt! Die Community unterstützt Sie bei der Verwendung.
\end{itemize}

\subsection{Arbeitsaufwand \& -aufteilung}
\begin{tabular} {| l | c | c | c | c |}
	\hline
	Task & Estimation & Sarah & Daniel & Team (Actual)	\\ \hline \hline
	Preparation and evaluation &  &  & 0.5h & 		\\ \hline
	Design &   &  &	2h & 			\\ \hline
	Implementation & &  & 10h &  \\ \hline
	Testing & 	& 	& 3h &  \\ \hline
	Documentation	&  &  & 2h & 	\\ \hline 
	Total	& 	&  & 22h &  \\
	\hline
\end{tabular}